\documentclass[12pt,a4paper,oneside,titlepage]{article}

\usepackage{fontspec}
\setmainfont{DejaVu Serif}
\setsansfont{DejaVu Sans}
\setmonofont{DejaVu Sans Mono}
\defaultfontfeatures{Ligatures=TeX}
\usepackage{polyglossia}
\usepackage[autostyle=true]{csquotes}
\setdefaultlanguage{russian}
\setotherlanguage{english}

\usepackage{listings}
\usepackage{graphicx}
\graphicspath{ {./} }

\lstset{
  columns=fullflexible,
  frame=single,
  breaklines=true,
}

\begin{document}

% НАЧАЛО ТИТУЛЬНОГО ЛИСТА
\begin{center}
  \hfill \break
  \textbf{
    \footnotesize{Министерство науки и высшего образования Российской Федерации}\\
    \hfill \break
    \footnotesize{Федеральное государственное бюджетное образовательное учреждение высшего образования}\\
    \small{«МОСКОВСКИЙ ГОСУДАРСТВЕННЫЙ ТЕХНИЧЕСКИЙ УНИВЕРСИТЕТ имени Н.Э.БАУМАНА\\(национальный исследовательский университет)»}}\\
\end{center}
\hfill \break
\normalsize{Факультет: Информатика и системы управления}\\
\hfill \break
\normalsize{Кафедра: Теоретическая информатика и компьютерные технологии}\\
\hfill\break
\begin{center}
  \textbf{\large{Лабораторная работа №1.2}}\\
  \large{Создание проекта Point\\
    по дисциплине «Языки и методы программирования»}\\
\end{center}
\hfill \break
\hfill \break
\hfill \break
\begin{flushright}
  \normalsize{
    Выполнил\\
    студент группы ИУ9-21Б\\
    Старовойтов А.И.\\
  }
  \normalsize{
    Проверил\\
    Посевин Д.П.
  }
\end{flushright}
\hfill \break
\hfill \break
\hfill \break
\hfill \break
\hfill \break
\hfill \break
\hfill \break
\hfill \break

\begin{center} Москва, 2022 \end{center}
\thispagestyle{empty} % выключаем отображение номера для этой страницы
% КОНЕЦ ТИТУЛЬНОГО ЛИСТА

\author{Старовойтов Александр}

\section{Цель работы}

Разобраться с совместимостью классов в Java на примере задачи Point

\section{Условия задачи}

\begin{itemize}
  \item
        Запустить новый проект
  \item
        Создать классы
  \item
        Запустить и проверить работоспособность
\end{itemize}

\section{Point}

\subsection{класс Main}
\begin{lstlisting}[language=java]
import com.company.Point;

public class Main {
    public static void main(String[] args) {

        Point PointA = new Point("A");
        System.out.println("Имя точки:"+PointA.getName());
        PointA.setCoord(1.0,1.0,1.0);
        System.out.println("Длинна радиус-вектора:"+PointA.getR());

        //PointA.n=10;
        PointA.val=100;
        System.out.println("ОбъемА:"+Point.val);

        Point PointB = new Point("B");
        System.out.println("ОбъемB:"+Point.val);

        System.out.println("Расстояние AB:"+Point.getDist(PointA, PointB));
    }
}
\end{lstlisting}

\subsection{класс Point}
\begin{lstlisting}
package com.company;

import static java.lang.Math.*;

public class Point
{

    private String name;
    private double x;
    private double y;
    private double z;
    private static int n;
    public static int val;

    public Point(String argName)
    {
        System.out.println("Запущен конструктор объекта com.company.Point");
        this.name = argName;
    }

    public String getName()
    {
        return name;
    }

    public void setCoord(double varX, double varY, double varZ)
    {
        this.x=varX;
        this.y=varY;
        this.z=varZ;
    }

    public double getR()
    {
        return pow(pow(this.x,2)+pow(this.y,2)+pow(this.z,2),0.5);
    }

    public static double getDist(Point a, Point b) {
        Point PointC = new Point("C");
        PointC.setCoord(a.x - b.x, a.y - b.y, a.z - b.z);
        return PointC.getR();
    }
}
\end{lstlisting}

\subsection{Вывод программы}
\begin{verbatim}
Запущен конструктор объекта com.company.Point
Имя точки:A
Длинна радиус-вектора:1.7320508075688772
ОбъемА:100
Запущен конструктор объекта com.company.Point
ОбъемB:100
Запущен конструктор объекта com.company.Point
Расстояние AB:1.7320508075688772
\end{verbatim}

\end{document}
